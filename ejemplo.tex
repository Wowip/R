# A minimal R Markdown example

A quote:

> Markdown is not LaTeX.

To compile me, run this in R:

    library(knitr)
    knit('001-minimal.Rmd')

See [output here](https://github.com/yihui/knitr-examples/blob/master/001-minimal.md).

## code chunks

A _paragraph_ here. A code chunk below (remember the three backticks):

```{r}
1+1
.4-.7+.3 # what? it is not zero!
```

## graphics

It is easy.

```{r}
plot(1:10)
hist(rnorm(1000))
```

## inline code

Yes I know the value of pi is `r pi`, and 2 times pi is `r 2*pi`.

## math

Sigh. You cannot live without math equations. OK, here we go: $\alpha+\beta=\gamma$. Note this is not supported by native markdown. You probably want to try RStudio, or at least the R package **markdown**, or the function `knitr::knit2html()`.

## nested code chunks

You can write code within other elements, e.g. a list

1. foo is good
    ```{r}
    strsplit('hello indented world', ' ')[[1]]
    ```
2. bar is better

Or inside blockquotes:

> Here is a quote, followed by a code chunk:
>
> ```{r}
> x = 1:10
> rev(x^2)
> ```

## conclusion

Nothing fancy. You are ready to go. When you become picky, go to the [knitr website](http://yihui.name/knitr/).

![knitr logo](http://yihui.name/knitr/images/knit-logo.png)

